\subsection{Broader exoplanet science}
\label{sec:broader_exoplanet_science}

\paragraph{Observe more transits over a long baseline to enable TTV measurements}
%Gravitational interactions between multiple bodies in a system with at least one transiting planet cause the planet's transit times to deviate from a linear ephemeris~\citep{agol_detecting_2005}.
Variations in transit times away from a linear ephemeris can be combined with planetary and orbital information derived from transits \textit{(a)} to confirm the planet-origin of transit signals and \textit{(b)} to determine the masses and system properties, for example mutual inclinations, of transiting and non-transiting planets~\citep{agol_detecting_2005}.

Our simulated dataset neglects orbital dynamics except to impose rule-of-thumb stability limits, assumes co-planarity between multiple planet systems, and assumes that multiple-planet system occurrence rates can be approximated as repeated draws from single-planet occurrence bins.
In short, it is insufficient to reliably predict TTV-detectability.
With that said, the best way to obtain useful TTV measurements is to observe as many high SNR transits as possible, over as long a baseline as possible.
Many TTV signals have timescales greater than a few years
%\todo[inline]{is there a nice formula? where does the $t^{5/2}$ scaling come from?}
and the information content in a TTV time series ideally scales as $t^{5/2}$~\citep{fabrycky_whitepaper_2013}.
Generally speaking then, the best extended mission scenario for measuring TTVs will include:
\begin{itemize}
	\item Long baseline observations of the same field, for instance one of \tesss continuous viewing zones.
	\item Observations of the \kepler field (for which measured transit times could resolve many TTV ambiguities from the 4-year dataset).
\end{itemize}
With these points in mind, \nhemi\ and \npole\ are likely the strongest scenarios for TTV detections over a single year's extended mission.

\begin{comment}
\textbf{Focus on compact multiple-planet systems}
\citet{muirhead_kepler-445_2015} estimate that $21^{+7}_{-5}\%$ of mid-M dwarfs host multiple planets with periods of less than 10 days.
\end{comment}



\paragraph{Targets beyond bright cool dwarf stars:}
\tesss main targets for the purpose of detecting small planets around bright stars are F-M dwarfs.
Observing other types of objects can enable broader exoplanet science.
For instance, we may wish to target:
\begin{itemize}
	\item open cluster members,
	\item known hosts of circumbinary planets,
	\item evolved stars (notably bright subgiants and red-giants for which it will be possible to detect asteroseismic oscillations, see below),
	\item all stars that are known to host transiting planets,
	\item all stars that are suspected to host \textit{any} planets (TOIs; KOIs from \kepler or \ktwo; candidates from \corot\!; candidates and confirmed planets from ground-based transit and RV surveys),
	\item white dwarfs,
	\item hot bright stars (OBA class),
	\item cool dwarf stars with spectral classes from late-M to mid-L,
	\item stars with well-measured properties from existing catalogs,
	%(for instance detailed abundance measurements, Hipparcos parallaxes, or very precise stellar parameters),

	\item eclipsing binaries and other multiple-star systems.
\end{itemize}
Some of these categories, notably asteroseismic targets, open cluster members, cool dwarfs, and circumbinary planets, have existing working groups dedicated to selecting targets and studying the resulting data once \tess begins its primary mission.
The \tess Target Star Selection team will incorporate target lists from these different working groups, while also incorporating many of the specific targets listed above\footnote{If the invested reader wishes to contribute to any of these lists, contact Joshua Pepper at \href{mailto:joshua.pepper@lehigh.edu}{joshua.pepper@lehigh.edu} or Keivan Stassun at \href{mailto:keivan.stassun@vanderbilt.edu}{keivan.stassun@vanderbilt.edu}} [Joshua Pepper, private communication].
\href{https://heasarc.gsfc.nasa.gov/docs/tess/}{Guest observer proposals} will likely also lead to allocation of pixels for many targets beyond bright F-M dwarf stars.

The relevance of all these objects to \tesss extended mission is that `observing time' (more precisely, data allocation) can be allotted differently between these targets in an extended mission than in the primary mission.
For instance, the benefits of obtaining precise stellar parameters and probing stellar physics via asteroseismology might merit a greater share of the data mass than is being proportioned for the primary mission.
The same could be argued for systems in which precise timing can greatly improve our ability to detect planets and characterize planetary systems.

We briefly highlight the cases for observing open clusters, circumbinary planets, asteroseismically-favorable targets, and stars with known or suspected planets below, and indicate which extended pointing scenarios we think will most benefit each observing program.

\begin{enumerate}
	\item \textit{Observing open clusters}.
	The main reasons to observe open clusters with \tess are (1) to discover planets in clusters (and then characterize them with other facilities) and (2) to perform stellar astrophysics relevant to exoplanets.
	Discovering planets in clusters will help determine their occurrence rate relative to field stars~\citep{meibom_same_2013}.
	This knowledge helps answer the question ``how well can small planets form and survive in dense clusters?''. 
	In terms of stellar astrophysics, an open cluster surveys also enables measurements of stellar rotation periods as a function of age (gyrochronology), which can constrain age-rotation-activity relations.
	Such a survey could also be used to identify eclipsing binaries in order to quantify their tidal evolution as a function of age.
	Soren Meibom is leading \tesss open cluster working group, which will construct lists of open cluster stars to be included in the \tess Candidate Target List.
	%It can also identify eclipsing binaries to quantify their stellar and tidal evolution.
	
	%The paragraph below based on the lists Meibom posted on the TESS wiki
	The majority of suitable open clusters with $V<16$ are near the galactic disk, and they are more common in the southern ecliptic hemisphere (with a south:north ratio of $\sim\!2:1$).
	While there are no known open clusters with $V<16$ at the north ecliptic pole's continuous viewing zone, there are a handful in that of the south ecliptic pole.
	Consequently, an extended mission scenario that observes the southern hemisphere, whether through the southern-conjugates of \nhemi\ and \npole, or through a scenario like \shemiAvoid, would be preferable for an open cluster survey.
	With that said, \npole\ would observe $\sim70$ open clusters with $V<16$, all within $42^\circ<\beta<68^\circ$.
	
	\item \textit{Circumbinary planets, circumprimary planets, and multiple star systems}.
	To date all $\sim$11 known circumbinary planets (CBPs) transit and are located in the \kepler field.
	Previous work has shown a statistical dearth of CBPs orbiting short-period ($P_\mathrm{bin} < 3$ day) eclipsing binaries~\citep{armstrong_abundance_2014,martin_planets_2014}, and formation requirements on such circumbinary planet systems are stronger than for planets orbiting EBs in which the EB has a relatively longer orbital period~\citep{martin_no_2015}.
	
	For order of magnitude purposes, consider two flavors of CBP detection from \tesss primary mission: those that are detected with $\ge3$ transits, and those with $\le 2$ transits.
	Assume that transiting CBPs will be mostly detected about eclipsing binaries (no CBPs to date are known to orbit a non-transiting eclipsing binary, although such objects likely exist in \keplers dataset \citep{martin_nontransiting_2014}).
	Given CBP orbital stability requirements, and the expectation that few CBPs will orbit $P_\mathrm{bin} < 3$ day eclipsing binaries, almost all of the $N_\mathrm{tra}\ge3$ CBP detections should happen in \tesss continuous viewing zones.
	Taking~\citetalias{Sullivan_2015}'s result that \tess will detect $\mathcal{O}(10^5)$ eclipsing binaries with $I<13$, and the fact that \tesss continuous viewing zones cover $2.2\%$ of the sky (908 deg$^2$), there will be roughly $2000$ EBs observed from one year over the primary mission.
	To date, \keplers Eclipsing Binary Working group has identified 2878 eclipsing and ellipsoidal binary systems of the 200,000 stars observed in \kepler\!'s 115 deg$^2$ field~\citep{kirk_keplerEB_2016}.
	This has led to $\sim$15 CBPs that are confirmed or in the process of being verified -- roughly a $0.5\%$ detection probability.
	Applying the same detection probability to \tesss CVZs, $\mathcal{O}(10)$ CBPs should be detected from 3 or more transits.
	
	For purposes of detecting CBPs from $\le 2$ transits, single-conjunction two-transit events will likely contribute to a major proportion of \tesss detections.
	Such events contain more information than single-star transits and can be used to make detections from a small number transits. 
	For instance, they enabled the confirmation of Kepler 1647b, the CBP with the longest known orbital period~\citep{kostov_kep1647b_2015}.
	Roughly 5 out of 1000 \kepler EBs with $P_\mathrm{bin}<30$ days had one-conjunction two-transit events [Haghighipour, private communication].
	Applying the same probability to \tess EBs means $\sim500$ \tess EBs will have these events.
	Perhaps 10\%-20\% will lead to CBP detections -- this means $\mathcal{O}(100)$ CBP detections from the \tess field outside of the CVZs.
	
	How will this affect the extended mission?
	The relative priority of these two different detection techniques -- robust CBP detections through $\ge 3$ transits in fields observed for a long time, vs. weak detections from $\le 2$ transits in fields observed for shorter durations -- merits further study.
	The former would advocate for a mission that maximizes the average observing time for all stars on a smaller sky area, for instance \npole.
	If this overlapped with the \kepler field, this would also enable \tess to detect transits of a subset of the \kepler CBPs.
	However the latter approach would argue for simply repeating the primary mission (\nhemi), or `event catching' single-conjunction two-transit events as in \hemis.
	
	Considering circum\textit{primary} planets, \tess should discover thousands of giant planets at orbital periods less than 10 days, predominantly in its full frame images, and with a heavy discovery bias towards the galactic disk (cf. Fig. 19 of ~\citetalias{Sullivan_2015}).
	Recent surveys have shown that roughly half of such hot Jupiter systems are expected to have stellar companions with semi-major axes between $50-2000\ \mathrm{AU}$~\citep{ngo_friends_2016}.
	The population of circumprimary planets that \tess will detect should be dominated by these systems.
	There should be a sufficiently large sample of these planets for follow-up imaging with adaptive optics to obtain CPP statistics, regardless of the extended mission.
	
	\item \textit{Asteroseismic targets}. See `Observing asteroseismic targets' paragraph in Sec.~\ref{par:asteroseismic_disc} below.
	
	
	\item \textit{Stars known or suspected to host transiting planets}. 
	If we know that the geometry of an exoplanetary system allows transits, the geometric bias against transit detection is removed.
	We should observe these targets \textit{(a)} to observe additional transits at precise times, enabling TTV searches, \textit{(b)} to obtain more precise photometry and thus refined parameters on known transiters, and \textit{(c)} in the case of ground-based detections, to observe companion transiting-planets that might not be detectable from the ground.
 	We discuss the prospects of \tess observing \kepler\!'s transiting planets in Sec.~\ref{par:TESS_as_followup}.
 	It will also be important to include the hosts of RV-detected planets in this sample. 
 	Although transit alignments are rare, combining transit and RV observables allows us to compute a planet's mean density, which is a necessary step towards detailed characterization. 
 	
 	For purposes of influencing an extended mission, the main bias in the currently known population of transiting planets is the \kepler field.
 	For an extended mission to follow up on the most currently-known transiting planets, it would be best target the northern ecliptic hemisphere, as in \npole\ or \nhemi.
	
	\item \textit{White dwarfs}.
	White dwarfs are scientifically interesting because of the window they offer into the long-term evolution of planetary systems.
	Their observational appeal for transit studies is that they have radii of $\mathcal{O}(R_\oplus)$, which means that transiting objects, whether planetary remnants (\textit{e.g.},~\citep{vanderburg_disintegrating_2015}), minor, or even major planets, produce large transit depths.
	However, the transit durations of major and minor planets are short ($T_\mathrm{dur} \sim R_\star / v_p$), and their transit probability is small ($\mathrm{prob}(\mathrm{tra}) \sim R_\star / a $).
	To date, no planets are known to transit white dwarfs (\citet{veras_postMS_2016} reviews the observational successes of the field along with this challenge).
	
	What this means for \tess is that a white dwarf survey would require short-cadence observations of many white dwarfs to overcome both the short transit durations and the low transit probability.
	While no group is currently advocating for this observing program, this may change by the time of an extended mission proposal.

\end{enumerate}

We do not discuss the observing cases for hot bright stars, late-M to mid-L dwarf stars, eclipsing binaries, or stars with well-measured properties.
% they can be made by say, GO observers

\begin{comment}
\paragraph{Microlensing survey}
Read relevant K2C9 papers. this is likely a bad idea (you can't get a good microlensing parallax like you could for K2C9)
\end{comment}