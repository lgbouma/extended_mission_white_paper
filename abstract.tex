\begin{abstract}
\label{sec:abstract}

The Transiting Exoplanet Survey Satellite (\textit{TESS}) will observe the southern and then the northern ecliptic hemispheres beginning in early 2018 and ending in 2020.
What then?
Our aim is to create a `time-capsule' of issues that we expect will be important when defining \textit{TESS'} long-term strategy.
We do this in two ways.
First, in this report we perform Monte Carlo simulations of the planets that \textit{TESS} detects in hypothetical extended missions.
By comparing `detected' planet populations, we explore the trade-offs between select scenarios.
Second, we create a discussion forum -- LINK -- where we initiate considerations outside of exoplanet detection statistics.
\textit{TESS} can and should be used to achieve broader science; this forum is a starting-place for community feedback in the process of defining \textit{TESS'} extended mission.

The quantitative results of this report are as follows:
for a third year's observing, provided \textit{TESS} satisfies basic engineering constraints, varying the direction the spacecraft points changes the relative number of newly detected, sub-Neptune radius planets by $\lesssim 30\%$.
In absolute terms, \textit{TESS} detects about the same number of sub-Neptune radius planets in a year of its extended mission as in a year of its primary mission: $\sim$1250.
The rate of planet-discovery at orbital periods $P > 20$ days roughly doubles in year 3 scenarios that either repeat the primary mission or focus towards an ecliptic pole.
We also show that the uncertainty of mid-transit time for \textit{TESS} objects of interest, in hours, is roughly $2\times$(the number of years after detection).
This means that in the long term, \textit{TESS} must re-observe the planets it finds during the primary mission, or else hinder follow-up characterization of its detected planets.
Observe a small number of extra transits with a few year's baseline improves knowledge of ephemerides by an order of magnitude.

% % PERHAPS MORE USEFUL EXTRAS:
%We find that varying where \tess observes on-sky
%(while pointing anti-Sun and avoiding terrestrial and lunar crossings) 
%changes the absolute number of newly detected planets by $\lesssim40\%$ over a third year of observing.
%Thus for most planet detection statistics, the largest qualitative difference between focusing on and away from the ecliptic with \tess may be the \textit{K2} fields.
%Given this point, our current bias is towards either the \nhemi or \npole scenarios, unless we can show that the K2-tess overlap in \elong would be worth it. \hemis seems too risky.
%In the long-term, \tess must re-observe the sky it sees over its first two years, or else lose knowledge of when its planets transit.
%These results and our discussion of the assumptions behind them inform the value engineering that will preempt a \tess extended mission proposal.

\end{abstract}
