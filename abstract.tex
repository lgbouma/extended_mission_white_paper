\begin{abstract}
\label{sec:abstract}

The Transiting Exoplanet Survey Satellite (\textit{TESS}) will observe the southern and then the northern ecliptic hemispheres beginning in early 2018 and ending in 2020. 
What then?
Performing Monte Carlo simulations of the planets that \textit{TESS} detects in hypothetical extended missions, we begin to explore the trade-offs between select observing strategies.
Notable results include that \textit{(1)} over a third year's extended mission, varying where \textit{TESS} looks on the sky changes the number of newly detected planets by $\lesssim30\%$ 
and that \textit{(2)} it will be possible to detect about as many new $P > 20$ day planets in one year of \textit{TESS}’s extended mission as in both years of the primary mission.
We also catalog many of the desires and opportunities which go beyond detecting small planets around bright stars, and recommend the next steps in defining \tesss extended mission.

% % PERHAPS MORE USEFUL EXTRAS:
%We find that varying where \tess observes on-sky
% %(while pointing anti-Sun and avoiding terrestrial and lunar crossings) 
%changes the absolute number of newly detected planets by $\lesssim40\%$ over a third year of observing.
%Thus for most planet detection statistics, the largest qualitative difference between focusing on and away from the ecliptic with \tess may be the \textit{K2} fields.
%Given this point, our current bias is towards either the \nhemi or \npole scenarios, unless we can show that the K2-tess overlap in \elong would be worth it. \hemis seems too risky.
%In the long-term, \tess must re-observe the sky it sees over its first two years, or else lose knowledge of when its planets transit.

\begin{comment}

%Context: 
\tess will conduct a two-year survey for transiting planets over 90\% of the sky. 
Once this survey is complete, in late 2019, what should the spacecraft do next? 
There do not appear to be any fundamental obstacles to continued operation for another 5-10 years. 

%Aims: 
(1) to define metrics by which to compare extended missions, (2) to evaluate how well any given \tess extended mission spacecraft pointing (for 1 year of extra observing) satisfies different technical requirements and science goals, and (3) to discuss the opportunities and risks that may come from each pointing strategy.

%Methods: 

%Results:
To first order, a single-year of extra observations with \tess can be imagined to double the number of transits observed about half of the primary mission's stars.
Assuming that our light curves are dominated by white noise, the phase-folded noise bins as the square-root of the number of transits, and thus the SNR for all these planets increases by roughly $\sqrt{2}$.
Our first important result is that there are many planets with SNRs insufficient for detection in the primary mission, but for which this extra $\sqrt{2}$ in their SNR will enable their detection: for all considered scenarios, there at least 1000 such planets with radii below that of Neptune (this assumes that the instruments age without catastrophic failure).
A related point is that it hardly matters where you look: for the different scenarios we considered, the number of newly detected planets varied by $<\pm 50\%$.
We comment on the possibility of optimizing for long-period planet detections, and find that given \tesss orbital constraints, a scenario in which the spacecraft centers its field on the ecliptic pole and advances every month is likely optimal (at least for one, and likely two, years of observing)


These results and our discussion of the assumptions behind them inform the value engineering that will preempt a \tess extended mission proposal.

\end{comment}

\end{abstract}
