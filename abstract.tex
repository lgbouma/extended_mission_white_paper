\begin{abstract}
\label{sec:abstract}

The Transiting Exoplanet Survey Satellite (\textit{TESS}) will observe the southern and then the northern ecliptic hemispheres beginning in early 2018 and ending in 2020.
What then?

This work creates `time-capsules' of issues relevant to \textit{TESS}'s long-term strategy, proceeding on two fronts.
First, in this report we perform Monte Carlo simulations of the planets that \textit{TESS} detects in hypothetical extended missions.
By comparing simulated populations of detected planets, we begin to explore trade-offs between observing scenarios.
Second, we create a forum (LINK!) to initiate discussion in topics beyond exoplanet detection statistics.
\textit{TESS} can and should be used across astronomy; this website is a starting-place for community feedback in the process of defining \textit{TESS}'s extended mission.

While many parameters define the spacecraft's observations (instrument cadence, target allocation strategy, total duration of observations), this report focuses on the issue of where to point the cameras on the sky.
We propose year-long scenarios that either observe mostly towards the ecliptic, mostly towards the ecliptic poles, or repeat the primary mission by overlapping at the poles while also observing slightly away from the ecliptic plane.

Our quantitative results are as follows: for a third year's observing, provided \textit{TESS} satisfies basic engineering constraints, varying the direction the spacecraft points changes the relative number of newly detected, sub-Neptune radius planets by $\lesssim 30\%$.
In absolute terms, \textit{TESS} detects about the same number of $R_p<4R_\oplus$ planets per year of its extended mission as per year of its primary mission ($\sim 1250$).
The detection rate of sub-Neptune radius planets at orbital periods greater than 20 days roughly doubles in certain year 3 scenarios.
While these results indicate that an extended \textit{TESS} mission will indeed be capable of detecting small planets at modestly longer periods, we caution that the uncertainty on mid-transit times for \textit{TESS} objects of interest, in hours, scales roughly as $2\times$(the number of years after detection).
In the long term, \textit{TESS} must re-observe the planets it finds during the primary mission to refresh knowledge of their ephemerides, or else hinder follow-up characterization in perpetuity.

\end{abstract}
