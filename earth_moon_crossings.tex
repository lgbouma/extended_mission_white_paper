\subsection{Earth/Moon crossings}
\label{sec:earth_moon_crossings}
\begin{figure}[!tb]
	%	\includegraphics{}
	\centering
	\includegraphics{figures/precision_memo.pdf}
	\caption{Relative precision in measured flux over a one hour integration time (scatter from contaminating background stars and PRF centroid offsets ignored). The noise sources described by~\citet{Sullivan_2015} are solid thin lines. The dashed lines exemplify noise from contaminating background flux, for instance from the Moon or Earth outside the \tess field, but still scattering light off  the \tess lens hood. A dynamical 3-body simulation lets us estimate the impact  lunar and Earth backgrounds have on \tesss photometry.
	The photometric precision with which we simulate observations has additional scatter about the thick black line owing to randomly-assigned contaminating stars.}
	\label{fig:noise_with_moon}
\end{figure}

When the Earth or Moon passes through \tesss camera fields they can
flood the CCD pixels to their full well capacity ($\sim2\times10^5$
photoelectrons).  Precision differential photometry becomes impossible
in any pixels that are directly hit during these crossings.  Even when
the body (the Earth or Moon) is not directly in the camera's field of
view, its light scatters off the interior of spacecraft's lens hood
and acts as a background source of contaminating flux across many of
the cameras\footnote{A detailed model for this process does not yet
  exist. It is intended that such a model will be developed during
  commissioning.}.  The Poisson noise in the number of photons
arriving from the Earth or the Moon in such a scenario degrades \tesss
photometric performance.  This can be important for field angles
$\theta \lesssim34^{\circ}$ (see Fig.~\ref{fig:lens_hood_suppression}
in Sec.~\ref{sec:appendix} of the appendix).  We show a few
representative background fluxes in Fig.~\ref{fig:noise_with_moon}.
The point of this figure is that for even modest background-counts the
effect of scattered light could severely reduce \tesss photometric
performance.  For instance when the moon is $30^\circ$ from the camera
boresights, corresponding (Fig.~\ref{fig:lens_hood_suppression}) to a
suppression of $10^{-5}$ from the lens hood on
$1.5\times10^6\mathrm{ct/s/px}$ from the moon, $15\mathrm{ct/s/px}$
reach the cameras.  This would correspond to a $2-3\times$ reduction
in photometric performance for $I_c < 11$.

\begin{figure}[!tb]
	\centering
	\includegraphics[angle=90,width=\textwidth]{figures/outage_earth_moon_primary.pdf}
	\caption{Fractional outage caused by Earth and Moon light either falling directly onto \tesss cameras or scattering off of their lens hoods and into the lenses, as a function of time in the orbit. The first year of observations are in the southern ecliptic hemisphere. The black dividing line indicates the beginning of `Year 2' of northern hemisphere observations. Note that the worst fractional outage per orbit is in Camera 1 (which points towards the ecliptic) over the first $\sim 5$ orbits of the second year.
	By `fractional outage' in these plots, we mean the fraction of target stars that could be observed with $\sigma_\mathrm{1hr} < 10^3\mathrm{ppm}$ precision that no longer can because of Earth or Moon light. 
	The time-step is 1/20th$^\mathrm{th}$ of an orbit.
	The plot has `spikes' because outage typically only occurs over a small fraction of the orbit.}
	\label{fig:earth_moon_primary}
\end{figure}
\begin{figure}[!tb]
	\centering
	\includegraphics[angle=90,width=\textwidth]{figures/outage_earth_moon_ecl_narrow_1yr.pdf}
	\caption{Same as Fig.~\protect\ref{fig:earth_moon_primary}, except for a hypothetical third year in which \tess observes the ecliptic with the cameras' long axis along the ecliptic plane. The latter half of the year experiences far less Earth and Moon interference than the first half. Considering it implausible that we would opt to sacrifice such a large fraction of our observing time ($\sim 50\%$ over the first 6 months), we study the \elong\:and \eshort\:scenarios instead, as they make useful observations during the first $\sim5$ months (shown in Fig.~\protect\ref{fig:proposed_pointings}).}
	\label{fig:earth_moon_elong}
\end{figure}

Separately from our planet detection simulation, we study the impact of these crossings in a dynamical simulation based on JPL NAIF's standard SPICE toolkit.
Given a nominal launch date, this code determines \tesss orbital phasing throughout its entire mission. 
At every time step of the three-body orbit, we calculate the distance between \tess and the other two bodies of interest, and the separation angles among each of the four cameras and each of the two bodies (eight angles in total). 
The gravitational dynamics behind this calculation treat the Earth, Moon, and Sun as point masses, and the \tess spacecraft as a massless test particle.
The spacecraft's inclination oscillates in the simulation as it will in reality.

Taking the Earth and Moon's integrated disk brightnesses as fixed values\footnote{$I_\moon = -13.5$, so the full moon delivers $1.5\times10^6 \text{ct/s/px}$, and the Earth delivers approximately $80\times$ that amount.} we use a model for scattered light suppression from the \tess lens hoods (Fig.~\ref{fig:lens_hood_suppression}), to tabulate the photon flux from each of these bodies onto each of the cameras throughout the orbit.

To evaluate the cumulative impact of these crossings on \tesss Primary and Extended Missions we ask: for each camera, what fraction of the total observing time is \tess unable to operate at desired photometric precision because of Earth and Moon crossings?
An upper limit for what we mean by `unable to operate at desired photometric precision' is when terrestrial or lunar flux make it impossible to observe any star in our selected target star catalog with photometric precision $<1 \text{mmag}$ over one hour of integration time.
This limit, $F_\text{max}$, is roughly
\begin{equation*}
 F_\mathrm{max} \approx 1000\mathrm{\ cts/s/px} = 200 F_\mathrm{readout\ noise}.
\end{equation*}

Considering \tesss precision (Fig.~\ref{fig:noise_with_moon}), as well as the target star catalog's apparent magnitude distribution (Fig.~\ref{fig:fig17_replica}), even a background of $100\ \text{ct/s/px}$ would be a problem, since it would hinder sub-mmag photometry for all stars with $I > 10$ ($\sim75\%$ of the target star catalog). We plot the percentage of target stars that are `lost' as a function of background counts (\textit{i.e.}, those that could be observed at sub-mmag precision over an hour, but no longer can) in Fig.~\ref{fig:outage_vs_background}.

A detailed model of how these crossings impact \tess photometry is outside the scope of this work.
%\todo[inline]{note that final precision depends on the length of window you integrate over. so resolving a 1mmag transit depth would be possible even for $\sigma_\text{1-hr} = 10^3$. Moreover, 1mmag maybe isn't even a good treshold. Everything with $I<14$ has $R<0.5R_\oplus$. Also depends on number of transits you see. So a reasonable thing would be like `with two transits, match limiting I mag   }
That said, to account at least qualitatively for this effect in our planet detection simulation we use a simple approximation: we impose that a camera has an `outage' if there are over $F_\text{thresh}\equiv300\ \text{cts/s/px}$ arriving from the Earth and Moon during a given exposure.
We then compute the average outage time per observing sector that \tess suffers in each of its cameras. 
For instance, a given sector might have $660$ hours of observing time over two spacecraft orbits, of which $220$ hours might either have the Earth, the Moon, or both shining with a background $F > F_\text{thresh}$. 
This would correspond to a fractional outage of $1/3$. 
We proceed by computing the mean of this fractional outage across all 13 sectors of a given year to derive a `mean camera outage' for each proposed pointing scenario.

\begin{table}[!tb]
	\centering
	\begin{tabular}{ | l | l | l | l | l | }
		\hline
		\ & Camera 1 & Camera 2 & Camera 3 & Camera 4 \\ \hline
		\multicolumn{1}{|c|}{Year 3 selected} & \  & \  & \  & \  \\ \hline
		\npole & 0 & 0 & 0 & 0 \\ \hline
		\nhemi & 2 & 1 & 0 & 0 \\ \hline
		\shemiAvoid & 2 & 0 & 0 & 0 \\ \hline
		\elong & 1 & 1 & 1 & 1 \\ \hline
		\eshort & 0 & 1 & 1 & 0 \\ \hline
		\hemis & 1 & 1 & 0 & 0 \\ \hline
		\multicolumn{1}{|c|}{Year 3 omitted} & \  & \  & \  & \  \\ \hline
		\texttt{pole\,(south)}  & 0 & 0 & 0 & 0 \\ \hline
		\texttt{hemi\,(south)} & 1 & 0 & 0 & 0 \\ \hline
		\texttt{hemi+ecl\,(south)} & 4 & 3 & 0 & 0 \\ \hline
		\texttt{ecl\_long\,(1yr)} & 4 & 4 & 4 & 4 \\ \hline
		\texttt{ecl\_short\,(1yr)} & 1 & 3 & 4 & 2 \\ \hline
		\texttt{allsky\,(poles)} & 0 & 0 & 0 & 0 \\ \hline
		\multicolumn{1}{|c|}{Primary mission} & \  & \  & \  & \  \\ \hline
		\texttt{hemi\,(south year 1)} & 2 & 1 & 0 & 0 \\ \hline
		\texttt{hemi\,(north year 2)} & 4 & 2 & 0 & 0 \\ \hline
	\end{tabular}
	\caption{Number of sectors (of 13 per year) `dropped' due to the Earth and Moon crossings in both selected and omitted Extended Missions, with those of the Primary Mission for reference. The method of `dropping' fields (which omits the temporal nature of the crossings, discussed in the text) gives a representative sense of the cumulative impact of Earth/Moon crossings. 
	Scenarios with \texttt{(south)} appended refer to their counterparts in the southern ecliptic hemisphere.
	\texttt{ecl\_long\,(1yr)} corresponds to a full year with the \tess field's long axis along the ecliptic, and \texttt{ecl\_short\,(1yr)} corresponds to the same, but with the long axis perpendicular to the ecliptic. These scenarios are neglected because their outages are time-correlated (see Fig.~\protect\ref{fig:earth_moon_elong}).
	\texttt{allsky\,(poles)} would be a scenario that observes in the manner of \hemis, but only on the ecliptic poles as in \npole.}
	\label{tab:dropped_fields}
\end{table}
As we mentioned in Sec.~\ref{sec:planet_detection_model}, our planet detection simulation is not explicitly time-resolved; it takes the ecliptic coordinates of camera fields for each orbit to compute the number of observations a given star receives over a specified mission.
We decide to approximate the effect of Earth and Moon crossings by selectively omitting the closest integer number of observing sectors corresponding to the `mean camera outage' described above.
For instance, if the `mean camera outage' was 17\% of \tesss observing time over a given year, we would omit the 2 (of 13) observing sectors that suffer the greatest number of lost hours, for that given camera.
The relevant number of omitted sectors is shown in Table~\ref{tab:dropped_fields}.
While this procedure ignores the temporal nature of the `outages' (which is shown resolved over time-steps of 1/20$^\mathrm{th}$ of an orbit in Figs.~\ref{fig:earth_moon_primary} and~\ref{fig:earth_moon_elong}), it gives a representative sense of the cumulative impact of Earth and Moon crossings over the course of a year.
We discuss the impact of this approximation for the Primary Mission in Sec.~\ref{sec:results_from_primary_missions}, and for Extended Missions in~\ref{sec:results_from_all_extended_missions}.
The summarized version is that modeling Earth/Moon crossings in this manner causes a drop of $<10\%$ of $R_p < 4R_\oplus$ planet detections compared to the case of not accounting for the crossings at all.
Given that Earth/Moon crossings typically last for a small fraction of an orbit (Fig.~\ref{fig:earth_moon_primary}), if the timescales required for the cameras to `re-settle' after the crossings are small compared to orbital timescales, then our approach may in fact over-estimate the effect's importance.
% I think this might be a bit better than it sounds because of how the earth/moon crossings are phased in the orbits. They're only happening for at worst ~1/2 of an orbit (see https://drive.google.com/folderview?id=0B2941jxrlPq0Z0tQZnRpd3pVOHc&usp=drive_web), but at roughly the same half across adjacent sectors. I.e. it's the same patches of sky, roughly, that aren't being observed across adjacent sectors, so just dropping the entire sector isn't as bad as you might think on first guess. It's admittedly somewhat sketch.

%A detailed model of this effect would take this dynamical simulation and incorporate it with the extant \tess instrument model to produce full simulated images of the \tess fields every 2 minutes over the course of a specified mission.
%Reduction to postage stamps and full frame images could be modeled in a `true' sense from these simulated images.
%The main apparatus of the planet detection simulation -- the process of creating planets, selecting transiting planets, and producing small images with the transiting planet signal injected, and computing a SNR for these  -- this is a somewhat long process. Maybe more like a masters thesis, and a phd thesis if you did it right.
