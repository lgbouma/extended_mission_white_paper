\subsection{Summary of Key Assumptions and Attributes of Planet Detection 
Simulations}
\label{sec:input_assumptions}

\begin{itemize}
  
	\item We focus almost exclusively on planets with $R_p < 4R_\oplus$.

        \item We assume the TRILEGAL catalog (modified to match
          interferometric radii, as described
          by~\citetalias{Sullivan_2015}) is an accurate representation
          of the stellar neighborhood to $\lesssim2\text{kpc}$.

        \item We omit the 5\% of the sky closest to the galactic disk
          (see Fig.~\ref{fig:positions_pointings}). We expect that
          \tesss large pixel size ($21\times21''$) combined with
          crowding near the galactic disk will cause substantial
          source confusion and a large astrophysical false positive
          rate in this area.  On a practical note, TRILEGAL cannot be
          queried within its run-time limit for some of these fields
          (cf.~\citetalias{Sullivan_2015} Sec 3.1).
        
	\item We assume prior knowledge of the radii and apparent
          magnitudes of TRILEGAL's synthetic stars, so we can
          prioritize the stars with a simple planet-detectability statistic
          (Eq.~\ref{eq:merit}).\footnote{Although it is difficult to determine
            accurate radii based on currently available data, we expect that by the time
            \tess launches \gaia will provide parallaxes and proper
            motions for many potential \tess targets, which will enable more reliable giant/dwarf discrimination and radius estimates.}.
          
	\item In evaluating a star's \texttt{Merit}
          (Eq.~\ref{eq:merit}), we compute the apparent magnitude of each "star"
          based on the sum of the flux from the star
          itself as well as any companion or background stars
          (whose presence will not be known {\it a priori}).
	            
	\item We use occurrence rates of planets as a function of radius
          and orbital period as calculated by~\citet{fressin_false_2013}
          and~\citet{dressing_occurrence_2015}.
          We assume these are accurate for the 
          $P \lesssim 180\ \text{day}$ planets to which \tess is 
          sensitive. We note in passing that~\citet{burke_terrestrial_2015} 
          find slightly higher occurrence rates for planets orbiting GK dwarfs 
          than~\citet{fressin_false_2013}.
          
	\item We assume the occurrence statistics of multiple-planet systems 	
		  can be approximated with repeated independent draws from the 
		  single-planet distributions. We further assume the orbits of planets 
		  in multiple planet systems to be 
		  coplanar and stable (with period ratios of at least 1.2 between 
		  adjacent planetary orbits).
          
	\item For our instrument and noise models, we assume:
	  \begin{itemize}
            
	  \item A point-spread function (PSF) derived from ray-tracing
            simulations.
            Compared with the PSF described by~\citetalias{Sullivan_2015} in 
            their Sec 6.1, ours incorporates lower charge diffusion (based on 
            laboratory measurements) and as-built (rather than ideal) optics.
            The net result is a slightly wider PSF, leading to $10\%$ fewer 
            $R_p<4R_\oplus$ planets than
            the same PSF model used by \citetalias{Sullivan_2015}.
            
          \item All stars are observed at the \textit{center} of the
            \tess CCDs. This ignores off-axis and chromatic
            aberrations within the \tess optics, and consequently
            ignores the angular dependence of the pixel response
            function (the fraction of light from a star that is
            collected by a given pixel).
            While~\citetalias{Sullivan_2015} attempted to model the
            field-angle dependence, we found some problems with this approach (see appendix
            Sec.~\ref{sec:changes_from_S15}).  
            Ignoring the field-angle dependence is a simplification that may 
            lead to loss of accuracy, but since this applies uniformly to all 
            the scenarios under consideration, it should still be possible
            to {\it compare} the results of different scenarios
            without much loss of accuracy.
            
         \item The time/frequency structure of all noise (except for
           stellar variability, see Eq.~\ref{eq:snr}) is `white'.  
           This means that we ignore time-correlated instrumental effects 
           such as spacecraft jitter, thermal fluctuations, and mechanical
           flexure, which we expect will be at least partly mitigated
           by the mission's data reduction pipeline.
                  
	 \item We assume the instruments perform equally well in year 3
           as in years 1 and 2.
           
	 \item The assumed contributors to white noise include: CCD
           read noise, shot noise from stars, a systematic noise floor
           of 60 $\mathrm{ppm}\cdot\mathrm{hr}^{1/2}$, and zodiacal
           background. See Fig.~\ref{fig:noise_with_moon} for the
           relative contributions of these terms as a function of
           apparent magnitude.
           
	\item The noise contributions from stellar intrinsic
          variability are assumed to be identical to those described
          by~\citetalias{Sullivan_2015} Sec3.5, which uses variability
          statistics from the \textit{Kepler} data computed
          by~\citet{basri_comparison_2013}.  Unlike all previously
          mentioned noise sources, we do not scale noise from stellar
          variability as $t_\mathrm{obs}^{-1/2}$, since the photon
          flux from stars may vary over time-scales similar to typical
          transit durations.  Instead, we assume the noise
          contribution from stellar variability is independent across
          transits, and thus scales as $N_\mathrm{tra}^{-1/2}$, for
          $N_\mathrm{tra}$ the number of observed transits.
	  \end{itemize}
          
	\item Our detection model is specified by Eq.~\ref{eq:detection_criterion}.
	Specifically,

	  \begin{itemize}
                
    \item We require $\geq$2 transits for detection.  We assume
	     the period can be recovered without ambiguity and likewise
	     there is no ambiguity in identifying which target star is
	     exhibiting a given transit signal.
		We also assume a step-function detection threshold: for 
	  $\mathrm{SNR_\mathrm{phase-folded}} > 7.3$ and $N_\mathrm{tra} \geq 2$, 
	  we rule transiting 
	  planets as detected, else they are not.
            
	\item The top $2\times 10^5$ $\mathtt{Merit}$-ranked targets
          (Eq.~\ref{eq:merit}) are observed at two-minute cadence, and
          the next $3.8\times10^6$ stars are observed at thirty-minute
          cadence.  We use~\citetalias{Sullivan_2015} Sec. 6.8
          approach to `blurring' transits with durations $\lesssim
          1\mathrm{hr}$, so that for longer cadence images shorter
          transits get shallower depths and longer apparent durations.
          As described in Sec.~\ref{sec:FFI_simulation}, we verify
          that under this assumption, our selection of PS stars is
          nearly complete for all detectable planets with
          $R_p<4R_\oplus$, and it is highly incomplete for Jupiter-sized planets.

          
	\end{itemize}

	\item We do not assume any prior knowledge of previous
          observations that may have been performed on our stars.  For
          instance, observing the ecliptic, we do not account for
          \tess\!-\ktwo overlap.
          
	\item For Earth and Moon crossings, we assume we can drop a
          fixed number of orbits of observing time for the cameras
          that suffer most from the Earth, the Moon, or both being in
          \tesss camera fields. We summarize this effect in
          Table~\ref{tab:dropped_fields}. Although this ignores the
          time-correlated nature of the outages shown in
          Figs.~\ref{fig:earth_moon_primary}
          and~\ref{fig:earth_moon_elong}, it is sufficient for
          comparing detected planet yields across missions, and
          leads to $\lesssim$10\% fewer $R_p<4R_\oplus$ planet 
          detections compared to the case of not accounting for the 
          crossings at all.
          
	\item We assume that we can (eventually) discriminate between
          astrophysical false positives (for instance background
          eclipsing binaries or hierarchical eclipsing binaries) and
          planet candidates.
          
	\item We can compute SNR with effective transit depth
          $\delta_\text{dil} = \delta \times D$ for
          $\delta=(R_p/R_\star)^2$ the transit depth, and
	  \begin{equation}
	D = \frac{\Gamma_T}{\Gamma_N + \Gamma_T}
	\label{eq:dilution}
	\end{equation}
	where $D$ is the dilution factor of a target star with
        incident photon flux $\Gamma_T$ from the target star and
        incident photon flux $\Gamma_N$ from neighboring stars 
        (\textit{i.e.}, background stars or binary companions).
	
	The noise is computed by creating a synthetic image for every
        host star with a planet that transits above a `pre-dilution'
        SNR threshold (this threshold is imposed for the sake of
        lowering our computational cost).  This $16\times16$ pixel
        image is of the number of photoelectrons \tess sees from the
        star and its companions/background stars at each pixel of each
        CCD.  We produce it through our PRF model, which in turn
        requires the host star's $T_\mathrm{eff}$ and apparent $I_c$.
        Over each image, the noise is computed for a range of possible
        aperture sizes about the brightest pixel
        (see~\citetalias{Sullivan_2015} Secs. 6.2 and 6.3), and then
        finally a single `noise' for each transit is selected by
        choosing the aperture size that minimizes the noise.

\end{itemize}
