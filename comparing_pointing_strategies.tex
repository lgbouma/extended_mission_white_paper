\subsection{Metrics by which we compare pointing strategies}
\label{sec:comparing_pointing_strategies}

We assess Extended Missions based on the risks and opportunities they
present, as well as through their performance on select technical and
science-based criteria.  These criteria are organized following an
approach originally outlined by~\citet{kepner_rational_1965}.
Summarizing them in list form:
\begin{description}
\item[Technical musts:] Cameras anti-sun? Solar panels collecting sunlight?
\item[Technical wants:] Duration of each sector $<28$ days? Earth/Moon crossings? Zodiacal background? Scattered sunlight off lens hood?
\item[Metrics in exoplanet science:]\
	\begin{itemize}
	\item number of newly detected planets (from 2-minute cadence fixed-aperture target stars, colloquially `postage stamps', as well as 30-minute full frame images -- both independently and combined); 
	\item number of new long-period planets (which may be detected through long period coverage, or by follow-up on single-transit events); 
	\item number of new habitable-zone planets; 
	\item number of new planets with ``characterizable'' atmospheres; 
	\item number of newly detected planets with bright host stars; 
	\item number of stars with transiting planets detected in the Primary Mission for which the Extended Mission reveals an additional
          transiting planet (usually long-$P$ companions to short-$P$ transiting planets);
        \item ability to improve transit ephemerides for previously detected transiting planets;
	\item ability to observe more transits over a longer baseline to enable searches for transit-timing variations.
        \end{itemize}
\end{description}

These metrics were chosen for their apparent importance as well as our
ability to quantify them with simulations. Of course there are other considerations
that may be very important but are more difficult to quantify:
\begin{itemize}
\item Prospects for altering target allocation weights between \{white
  dwarfs, known planet-hosts, candidate planet-hosts, circumbinary \&
  circumprimary planets, open clusters, evolved stars (notably to
  detect asteroseismic oscillations), dwarf stars later than M7, stars
  with well-measured properties, $\ldots$\}.
\item Prospects for observations relevant to stellar astrophysics, many of which may overlap with exoplanetary science.
    For instance, we may wish to try and measure a large sample of stellar rotation periods, or allocate a larger fraction of the data mass for short-cadence asteroseismic targets.
    We may also wish to observe optical/near-IR variable targets across the sky, in particular 
	pulsating stars (Cepheids, RR Lyrae, Delta Scuti, slowly pulsating B stars),
	eruptive stars (protostars, giants, eruptive binaries, flare stars), 
	cataclysmic variables (dwarf novae, novae, supernovae), 
	rotating variable stars (deformed by ellipsoidal variations, showing variability from stellar spots or magnetic fields),
	and eclipsing binaries.
	Long-term observations of starspot modulation can be used to characterize long timescale stellar activity cycles.
\item Prospects for solar-system science, such as observeations of main belt asteroids and the brightest near Earth asteroids.
\item Prospects for extragalactic astronomy and high energy astrophysics; for instance, gathering light curves of variable active galactic nuclei.
\end{itemize}

Regarding opportunities and risks, the following need to be considered:
\begin{description}
	\item[Opportunities:]\ 
	\begin{itemize}
	\item What's best on a $>1$ year horizon for planet detections? For instance, if it were known in advance
	that \tess would continue operations for several additional years (or even 10
	years), would such knowledge affect the optimal choice of the immediate
	one-year plan?
		\item Ability to move targets detected in FFIs to PSs in Extended Mission
		\item Shorten the cadence of FFIs \& lengthen the cadence of `target' stars.
		\item \textit{\tess as follow-up mission:} ability to
                  observe \corot objects; ability to observe \kepler
                  field (key benefits in broader science wants above);
                  ability to observe \ktwo fields (follow-up \ktwo
                  few-transit objects); ability to observe targets
                  previously monitored by ground-based surveys.
		\item \textit{Follow-up for \tess\!:} 
		potential for \jwst follow-up? 
		Potential for \cheops follow-up? 
		Ability to get \tess photometry contemporaneously with ground-based observations?
		Ability to observe from both North and South hemispheres on Earth?
		\item Impact on Guest Investigator program?
	\end{itemize}
	
	\item[Risks:] 
	Risk of spacecraft damage? 
	Risk of not meeting threshold science (to be defined)? 
	Risk of excessive false positives, for instance from crowding? 
	Would partial instrument failure in Primary Mission make this scenario infeasible? 
	Would reduced precision (from aged CCDs, worse pointing accuracy, or other mechanical sources) invalidate this scenario? 
	Risk of planet detection simulation over or under-estimating planet yield?
\end{description}
