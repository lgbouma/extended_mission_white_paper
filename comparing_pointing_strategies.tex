\subsection{Metrics by which we compare pointing strategies}
\label{sec:comparing_pointing_strategies}

We assess extended missions based on the risks and opportunities they present, as well as through their performance on select technical and science-based criteria.
These criteria are organized following an approach originally outlined in~\citet{kepner_rational_1965}.
Summarizing them in list-form:
\begin{description}
	\item[Technical musts:] Cameras anti-sun? Solar panels collecting sunlight?
	\item[Technical wants:] Duration of each sector $<28$ days? Earth/Moon crossings? Zodiacal background? Scattered sunlight off lens hood?
	\item[Exoplanet science wants:]\
	\begin{itemize}
		\item \textit{New planet detection statistics:} 
		number of newly detected planets (considering postage stamps and full frame images, independently and combined); 
		number of new long-period planets (which may be detected through long period coverage, or by follow-up on single-transit events); 
		number of new habitable zone planets; 
		number of new planets with characterizable atmospheres; 
		number of newly detected planets with bright host stars; 
		number of systems with extra detectable planets (usually long-$P$ companions to short-$P$ transiting planets);
		number of newly detected planets with bright host stars.
		\item \textit{Broader TESS science:}
		better ephemeris times on known TOIs;
		observe more transits over a long baseline to enable TTV searches; 
		ability to alter target allocation weights between \{white dwarfs, known planet-hosts, candidate planet-hosts, circumbinary \& circumprimary planets, open clusters, evolved stars (notably to detect asteroseismic oscillations), dwarf stars later than M7, stars with well-measured properties, $\ldots$\}.
	\end{itemize}
	\item[Broader science wants:]
	Make observations relevant to stellar astrophysics, many of which may overlap with `exoplanet science wants'. For instance, observe many stars to measure stellar rotation periods, or allocate a larger fraction of the data mass for short-cadence asteroseismic targets.
	May also wish to observe optical/near-IR variable targets across the sky, in particular 
	pulsating stars (Cepheids, RR Lyrae, Delta Scuti, slowly pulsating B stars),
	eruptive stars (protostars, giants, eruptive binaries, flare stars), 
	cataclysmic variables (dwarf novae, novae, supernovae), 
	rotating variable stars (deformed by ellipsoidal variations, showing variability from stellar spots or magnetic fields),
	and eclipsing binaries.
	In particular, star spots can be used as indicators for magnetic activity to characterize long timescale stellar activity.
	In solar system science, may wish to observe main belt asteroids and the brightest near Earth asteroids.
	In galactic astronomy and high energy astrophysics, may wish to gather light curves for variable active galactic nuclei.

	\item[Opportunities:]\ 
	\begin{itemize}
		\item What's best on a $>1$ year horizon for planet detections? For instance, if \tess were to continue for 5, or even 10 years, would any strategy be optimal?
		\item Ability to move targets detected in FFIs to PSs in extended mission
		\item Shorten the cadence of FFIs \& lengthen the cadence of `target' stars.
		\item \textit{\tess as follow-up mission:} 
		ability to observe \corot objects; 
		ability to observe \kepler field (key benefits in broader science wants above); 
		ability to observe \ktwo fields (follow-up \ktwo few-transit objects);
		ability to observe targets previously monitored by ground-based surveys.
		\item \textit{Follow-up for \tess:} 
		potential for \jwst follow-up? 
		Potential for \cheops follow-up? 
		Ability to get \tess photometry contemporaneously with ground-based observations?
		Ability to observe from both North and South hemispheres on Earth?
		\item Impact on Guest Investigator program?
	\end{itemize}
	
	\item[Risks:] 
	Risk of spacecraft damage? 
	Risk of not meeting threshold science (to be defined)? 
	Risk of excessive false positives, for instance from crowding? 
	Would partial instrument failure in primary mission make this scenario infeasible? 
	Would reduced precision (from aged CCDs, worse pointing accuracy, or other mechanical sources) invalidate this scenario? 
	Risk of planet detection simulation over or under-estimating planet yield?
\end{description}

The specific weights between all of these desires will be allocated, explicitly or implicitly, in the process of defining an extended mission.
This report is structured to address technical musts and wants in Sec.~\ref{sec:approach}, new exoplanet detection statistics in Sec.~\ref{sec:newly_detected_planet_metrics}, broader science \tess could perform beyond planet detection, whether or exoplanet-related or not, in Sec.~\ref{sec:broader_science_beyond_planet_detection_stats}, and the remaining opportunities and risks in Sec.~\ref{sec:risks_opps}.
%We conclude and present recommendations for further consideration in Sec.~\ref{sec:conclusions}.