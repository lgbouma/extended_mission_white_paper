\documentclass{article}
\usepackage{graphicx}
\usepackage{amssymb}
\usepackage{amsmath}
\begin{document}

Let $s(\theta)$ be the lens hood suppression as a function of angle $\theta$ 
from the camera boresight to a given point-source on the sky. By definition,
\begin{equation}
s(\theta) \equiv \frac{F_\mathrm{obs}}{F_\mathrm{ns}},
\end{equation}
for $F_\mathrm{obs}$ the observed flux of the source (that which reaches the 
CCD), and $F_\mathrm{ns}$ the flux that would be observed with no 
suppression.
The observed flux of the source can then be written as a function of $\theta$
\begin{align}
	F_\mathrm{obs}(\theta) &= s(\theta) F_\mathrm{ns}  \\
	&= s(\theta) F_0 10^{-0.4(m_\mathrm{ns} - m_0)} ,
\end{align}
for $F_0$ the (non-suppressed) flux corresponding to a source with zero-point 
apparent magnitude $m_0$, and $m_\mathrm{ns}$ the apparent magnitude of 
the source with no suppression.
Winn 2013 tabulates $F_0 = 1.6\times10^6\,\mathrm{ph/s/cm^2}$ 
for an $I=0$, G2V star.
Thus
\begin{equation}
F_\mathrm{obs}(\theta) = (1.6\times 10^6) s(\theta) 10^{-0.4 m_\mathrm{ns}} 
\quad \mathrm{[ph/s/cm^2]}.
\end{equation}

We can then write the following expression for $\mu\equiv F_\mathrm{obs} A \eta 
/ N$, the mean incident flux on the camera of interest:
\begin{equation}
\mu =  (1.6\times 10^6) s(\theta) 10^{-0.4 m_\mathrm{ns}} \frac{A 
\eta}{N} \quad \mathrm{[ct/px/s]},
\end{equation}
for $A$ the effective observing area in $\mathrm{cm^2}$, $N$ the 
number of pixels per camera, and $\eta$ the quantum efficiency.
For TESS, $A=69.1\,\mathrm{cm^2}$, $N=4096^2$, and $\eta\approx 1$. Plugging in 
these numbers gives
\begin{equation}
\mu = 6.59  s(\theta) 10^{-0.4 m_\mathrm{ns}}\quad 
\mathrm{[ct/px/s]}.
\end{equation}

Note that the above expression assumes that the scattered flux from the source 
is uniformly spread across the CCD. The reality may be quite different.
In addition, the zero-point we used relied on an $I$ magnitude calibration -- 
for accuracy, separate zero-points based on different bandpasses should be 
computed.
Assuming a Poisson arrival rate, the standard deviation in the number of counts 
per pixel per 2 second readout is then
\begin{equation}
\sigma \approx \mu^{1/2} = \left[ 13.2 s(\theta) 10^{-0.4 m_\mathrm{ns}} 
\right]^{1/2}\quad \mathrm{[ct/px\ RMS\ per\ 2\ sec\ image]}.
\label{eq:added_RMS}
\end{equation}
To compute the net effect on TESS's noise budget, add $\sigma$ from 
Eq.~\ref{eq:added_RMS} in quadrature with Eq. BLAH.


\end{document}