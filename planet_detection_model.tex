\subsection{Description of Planet Detection Model}
\label{sec:planet_detection_model}

\citet{Sullivan_2015} (hereafter, \citetalias{Sullivan_2015})
developed a simulation of \tesss planet and false positive detections
based on the spacecraft and payload design specified by
\citet{ricker_transiting_2014}.  We adapt this simulation for Extended
Mission planning.  With our additions, we can change where \tess looks
in additional years of observing while holding fixed all other
mission-defining parameters.  Our approach is to run our planet
detection simulation for each plausible pointing strategy, and to
compare the relative yields of detected planets.  This lets us compare
Extended Mission scenarios with one another and with the Primary
Mission.

\paragraph{Background on synthetic catalogs:}

\tess will be sensitive to sub-Neptune sized transiting planets orbiting M
dwarfs out to $\sim\!200\,\text{pc}$ and G dwarfs out to
$\sim\!1\,\text{kpc}$~(\citetalias{Sullivan_2015}, Sec. 2.3).  It will be
sensitive to giant planets and eclipsing binaries across a significant
fraction of the galactic disk.  With this sensitivity in mind, the
stellar catalog we `observe' in our planet detection simulation is
drawn from the output of TRILEGAL, a population synthesis code for the
Milky Way~\citep{girardi_star_2005}.  ~\citetalias{Sullivan_2015} made
some modifications to the catalog, notably in the M dwarf
radius-luminosity relation, to better approximate interferometric
stellar radii measurements.  We retain these modifications; the modified
TRILEGAL stellar catalog shows acceptable agreement with
observations\footnote{Looking closely at the radius-luminosity
  relations, we do see non-physical interpolation artifacts. These
  outliers are visible in Figs.~\ref{fig:fig17_replica}
  and~\ref{fig:fig17_radius_on_x} below, but are a small enough subset
  of the population that they can be ignored on statistical grounds in this 
  work.}, specifically
the Hipparcos
sample~\citep{perryman_hipparcos_1997,van_leeuwen_validation_2007} and
the $10\text{pc}$ RECONS sample~\citep{henry_solar_2006}.

With a stellar catalog defined, we populate the stars in the catalog
with planets based on occurrence rates derived from the
\textit{Kepler} sample.  We use rates~\citet{fressin_false_2013} found
for planets orbiting stars with $T_\text{eff} > 4000\text{K}$ and
those that~\citet{dressing_occurrence_2015} found for the remaining M
and late K dwarfs.

\paragraph{Detection process:}
We then simulate transits of these planets.  Assuming the transit
depth and number of transits are known, we use a model of \tesss point
spread function (PSF) to determine optimal photometric aperture sizes
for each postage stamp star (\textit{i.e.,} we compute the noise for
all plausible aperture sizes, and find the number of pixels that
minimizes this noise).  With the aperture sizes and noise
corresponding to a given integration time known, we compute a signal
to noise ratio for each transiting object.  Our model for planet
detectability is a simple step function in SNR: if we have two or more
transits and $\text{SNR} > 7.3$, the planet is `detected', otherwise
it is not detected\footnote{The value of this threshold is chosen to
  ensure that no more than one statistical false positive is present in
  a pipeline search of $2\times10^5$ target stars. When observing a
  greater number of stars, for instance in the full frame images, 
  a higher threshold value should be imposed to maintain the same condition. We
  discuss this further in Sec.~\protect\ref{sec:risks_caveats}.}. 
Cast as an equation,
\begin{equation}
\mathrm{Prob(detection)}=\begin{cases}
1 \quad \mathrm{if\ SNR_{phase-folded}} > 7.3\ \mathrm{and}\ 
	N_\mathrm{tra} \geq 2 \\
0 \quad \mathrm{otherwise},
\end{cases}
\label{eq:detection_criterion}
\end{equation}
for $N_\mathrm{tra}$ the number of observed transits and 
$\mathrm{SNR_{phase-folded}}$ the phase-folded signal to noise ratio, defined 
in Eq.~\ref{eq:snr} below.
Our model for \tesss photometric precision is described 
by~\citetalias{Sullivan_2015} and also shown in
Fig.~\ref{fig:noise_with_moon}.

\paragraph{Assumptions of SNR calculation:}
A realistic simulation of the signal detection process
would begin with a realistic noise model for each 2-second CCD readout, taking into account both
instrumental and astrophysical variations. These 2-second data would be stacked into 2-minute
averages for the PSs and 30-minute averages for the FFIs. The time series would 
then be prepared and searched for transit signals using the same code that is 
used on real data.

Our calculation is much simpler. We calculate the phase-folded SNR based on an 
analytic approximation.
We implicitly assume that the correct period can be identified and that the noise follows
the simple model of \citetalias{Sullivan_2015}, which includes instrumental noise
and an empirically-determined level of stellar variability.
We also assume the transit signal to be constant in amplitude, although it may 
be reduced
in amplitude (``diluted'') below $(R_p/R_\star)^2$ due to the constant light 
from other stars
that fall within the same photometric aperture.
We then simply tally the number of \tess fields a
given host star falls in, which corresponds to a known total
observing baseline.  Knowing a planet's orbital period, and assuming the
planet is randomly phased in a circular orbit when \tess begins observing, 
we can count the number of transits \tess observes for the planet over the 
total baseline.
Using a model PSF, we determine ideal aperture sizes and obtain 
an estimated noise per transit.
Summarizing the relevant terms in an equation,
\begin{align}
    \mathrm{SNR}_\mathrm{phase-folded} &\approx
	\sqrt{N_\mathrm{tra}} \times \mathrm{SNR}_\mathrm{per-transit}\nonumber \\
	 &= \sqrt{N_\mathrm{tra}} \times 
	\frac{\delta \cdot D}{\left(\frac{\sigma_\mathrm{1hr}^2}{T_\mathrm{dur}} 
		+ \sigma_v^2 \right)^{1/2}}, 
	\label{eq:snr} 
\end{align}
for $\delta$ the undiluted transit depth; $D$ the dilution factor
computed from background and binary contamination (Eq.~\ref{eq:dilution});
$\sigma_\mathrm{1hr}$ the summed noise contribution from CCD read noise, 
photon-counting noise from the star, a systematic 
$60\,\mathrm{ppm\cdot hr^{1/2}}$ noise floor, and zodiacal noise;
$T_\mathrm{dur}$ the transit duration in hours, and $\sigma_v$ the intrinsic
stellar variability (cf.~\citetalias{Sullivan_2015} Sec 3.5).
Note that our only noise contribution which is not white comes from stellar 
variability, 
which following~\citetalias{Sullivan_2015} we assume to be independent over 
transits and also independent of the duration of a given transit.
Finally, we multiply the SNR per transit by the square root of the number of transits
observed.

We have changed some aspects of this simulation
since the work by~\citetalias{Sullivan_2015} was published. These
changes are described in Sec.~\ref{sec:changes_from_S15}.
