\subsection{Description of planet detection model}
\label{sec:planet_detection_model}

\citet{Sullivan_2015} (hereafter, \citetalias{Sullivan_2015})
developed a simulation of \tesss planet and false positive detections
based on the spacecraft and payload design specified by
\citet{ricker_transiting_2014}.  We adapt this simulation for extended
mission planning.  With our additions, we can change where \tess looks
in additional years of observing while holding fixed all other
mission-defining parameters.  Our approach is then to run our planet
detection simulation for each plausible pointing strategy, and to
compare the relative yields of detected planets.  This lets us compare
Extended Mission scenarios with one another and with the Primary
Mission.

\paragraph{Background on synthetic catalogs:}

\tess is sensitive to sub-Neptune sized transiting planets orbiting M
dwarfs out to $\lesssim200\text{pc}$ and G dwarfs out to
$\lesssim1\text{kpc}$~(\citetalias{Sullivan_2015}, Sec. 2.3).  It is
sensitive to giant planets and eclipsing binaries across a significant
fraction of the galactic disk.  With this sensitivity in mind, the
stellar catalog we `observe' in our planet detection simulation is
drawn from the output of TRILEGAL, a population synthesis code for the
Milky Way~\citep{girardi_star_2005}.  ~\citetalias{Sullivan_2015} made
some modifications to the catalog, notably in the M dwarf
radius-luminosity relation, to better approximate interferometric
stellar radii measurements.  We retain these modifications; the modified
TRILEGAL stellar catalog shows acceptable agreement with
observations\footnote{Looking closely at the radius-luminosity
  relations, we do see non-physical interpolation artifacts. These
  outliers are visible in Figs.~\ref{fig:fig17_replica}
  and~\ref{fig:fig17_radius_on_x} below, but are a small enough subset
  of the population that we ignore them for this work.}, specifically
the Hipparcos
sample~\citep{perryman_hipparcos_1997,van_leeuwen_validation_2007} and
the $10\text{pc}$ RECONS sample~\citep{henry_solar_2006}.

With a stellar catalog defined, we populate the stars in the catalog
with planets based on occurrence rates derived from the
\textit{Kepler} sample.  We use rates~\citet{fressin_false_2013} found
for planets orbiting stars with $T_\text{eff} > 4000\text{K}$ and
those that~\citet{dressing_occurrence_2015} found for the remaining M
and late K dwarfs.

\paragraph{Detection process:}
We then simulate transits of these planets.  Assuming the transit
depth and number of transits are known, we use a model of \tesss point
spread function (PSF) to determine optimal photometric aperture sizes
for each postage stamp star (\textit{i.e.,} we compute the noise for
all plausible aperture sizes, and find the number of pixels that
minimizes this noise.  With the aperture sizes and noise
corresponding to a given integration time known, we compute a signal
to noise ratio for each transiting object.  Our model for planet
detectability is a simple step function in SNR: if we have two or more
transits and $\text{SNR} > 7.3$, we rule it as `detected', otherwise
it is not detected\footnote{The value of this threshold is chosen to
  ensure that no more than 1 statistical false positive is present in
  the `detections' from $2\times10^5$ target stars. Observing a
  greater number of stars, for instance in full frame images, should
  require a higher threshold value to maintain the same condition. We
  discuss this in Sec.~\protect\ref{sec:risks}.}.  Our model for
\tesss photometric precision is described
by~\citetalias{Sullivan_2015} and shown in
Fig.~\ref{fig:noise_with_moon}.

\paragraph{Assumptions of SNR calculation:}
Our approach to computing SNRs for each transiting object is not
time-resolved.  In other words, we are not simulating every 2 second
CCD readout, stacking those hypothetical readouts into 2 minute
cadence postage stamps and 30 minute full frames, and then reducing
simulated light curves.

Our calculation is simpler.  We assume perfect period-recovery, phase
folding, and identical conditions between transits.  We also assume
that we observe a constant transit depth, which is diluted by binary
companions and background stars in the same manner between transits.
Our approach is then to simply tally the number of \tess fields a
given host star falls within, which corresponds to a known total
observing baseline.  Assuming random orbital phasing, we then compute
the number of transits \tess observes for planets of any given host.

With a model PSF, we determine ideal aperture sizes (see two
paragraphs above), and then obtain an accurate noise per transit
(since the transit durations are known, and we assume our noise,
computed first over a single hour, then bins like white-noise,
\textit{i.e.,} proportionate to the inverse square root of the time
in-transit).

Coupled with the known transit signal, this gives us the SNR per
transit, and then to `phase-fold our light-curves' (light-curves which
are never explicitly computed point-by-point) we just\footnote{We
  actually take a quadrature sum of both transit and occultation
  signals, but this is negligible for planets. It only matters for the
  case of eclipsing binaries, which we ignore in this work.} multiply
the SNR per transit by the square root of the number of transits
observed.
 
We have changed other aspects of this simulation
since~\citetalias{Sullivan_2015} was published, and describe these
changes in Sec.~\ref{sec:changes_from_S15} of the appendix.
