\section{Changes from~\citet{Sullivan_2015}}
\label{sec:changes_from_S15}
%\setcounter{page}{1}
%\renewcommand{\thepage}{\thesection -\arabic{page}}
\paragraph{On the angular dependence of \tesss pixel response function:} 
In our SNR calculation, we do not keep track of individual times for every transit. 
How then do we assign PSFs to different transits from the same object that fall on different regions of the CCDs, and thus should have slightly better or degraded PSFs? (Largest cumulative flux fraction at the CCD's center, smallest at the corners). 

~\citetalias{Sullivan_2015} dealt with this by computing the mean of all the field angles (distance from the center of the CCD axis), and then passing this mean into a look-up table for PSFs based on four PSFs that had been computed from a ray-tracing model at four different field angles.
~\citetalias{Sullivan_2015} then `observes' each eclipsing object with a single class of PSF.
This leads to the implausible phenomenon that extra observations can actually \textit{lower} the SNR of an eclipsing object if they are taken with an unfavorable field angle/PSF. 
This effect is largely off-set by the extra pointing increasing the SNR, but for extended missions (coming back to the same objects at potentially very different field angles) it winds up reducing observed SNR for $\sim$3\% of detected objects.

In extended missions (as well as in the primary mission) we expect stars to land on very different regions of the CCD over the course of being observed.
We simplify this in our work by assuming that all stars land on the center of the \tess CCDs (the green curve of~\citetalias{Sullivan_2015} Fig. 13). 
This assumption is justified because our chief point of quantitative comparison is the ability of different pointing strategies to impact \tesss planet yield in extended missions, and there is little \textit{a priori} reason to assume that any one pointing scenario should be biased for an extra amount of stars to land on the `bad regions' of \tesss CCDs.
%We also note that the PRF performance only decreases noticeably for $1 - \pi(12^\circ)^2 / (24^\circ)^2 \approx 21\%$ of the pixel area, and this modification has the benefit of omitting the rare ``extra-observations lead to reduced SNR'' effect described in the above paragraph.
	
