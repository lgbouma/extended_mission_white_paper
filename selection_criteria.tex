\subsection{Selecting Target Stars (and Modelling Full Frame Images)}
\label{sec:selection_criteria}
For the Primary Mission, \tesss short cadence (2 min) targets will be drawn from a subset of the \tic. 
The exact manner in which these targets will be chosen has not yet been determined.

We know that for \tess to detect small transiting planets
it should observe stars that are small and bright.  For this work, we
define a simple statistic, \texttt{Merit}, proportional to the SNR we
should expect from a given planet orbiting the star:
\begin{align}
\texttt{Merit} &\equiv 
	\frac{1/R_\star^2}{\sigma_\text{1-hr}(I_c)/\sqrt{N_\text{obs}}}\ ,
\label{eq:merit}
\end{align}
where $R_\star$ is the radius of the star in question,
$\sigma_\text{1-hr}$ is the relative precision in flux measurements
over one hour of integration time, taken from an empirical fit to
Fig~\ref{fig:noise_with_moon}, $I_c$ is the Cousins band $I$ magnitude
\tess observes for the star (or more precisely, the star system) and
$N_\text{obs}$ is the number of observations the star receives over
the course of the mission.  For multiple star systems, $R_\star$ is
taken to be the radius of
the planet-hosting star, and the $I_c$ magnitude is computed from the 
combined flux of all the stars.

\begin{figure*}[!tb] %[!thb]
	\centering
	\includegraphics{figures/positions_pointings_kepler.pdf}
	\caption{Selected target stars in the Primary Mission. Their surface 
	density increases as $\sqrt{N_\text{obs}}$ towards the poles because of the 
	weight used	in Eq.~\protect\ref{eq:merit}. Gaps in the focal plane array 
	between each of a 
	given camera's four CCDs leads to the slight deviations from 
	continuous observing at the ecliptic pole.}
	\label{fig:positions_pointings}
\end{figure*}
\begin{figure}[!tb] %[!thb]
	\centering
	\includegraphics{figures/fig17_replica.pdf}
	\label{fig:fig17_replica}
	\caption{Target star Cousins I magnitude against effective temperature 
		(replicates Figure 17 of~\protect\citetalias{Sullivan_2015}). 
		Target stars are selected as the best $2\times10^5$ stars according to $\texttt{Merit}\equiv \sqrt{N_\text{obs}}(1/R_\star^2)/\sigma_\text{1-hr}(I_c)$. 
		The top subplot shows 1 in 10 stars. This simple model could inform the target selection to be performed on the \tic. 
		The lower histogram is bimodal, selecting heavily for M
		dwarfs, and selecting more F and G dwarfs than K dwarfs. This
		shape arises from the combined $1/R_\star^2$ and
		$1/\sigma_\text{1-hr}(I_c)$ weights: the fact that the minimum
		falls between FG and M stars is because there are many more FG stars than K stars in our
		catalog, and because M stars offer the largest signals. Note also the
		dip in the TRILEGAL (\& observed) $V$-band luminosity functions for K stars
		(see~\citetalias{Sullivan_2015} Figure 5).
		% N.b. Winn showed this bias ANALYTICALLY in the searchable stars memo (he showed there are ~4* more stars e.g. in 0.75<R/Rsun<0.85 than 0.5<R/Rsun<0.6)
		
		The outliers visible in the upper scatter plot represent nonphysical "stars",
		possibly artifacts of the Padova-to-Dartmouth conversion performed by~\citetalias{Sullivan_2015}.
		These comprise fewer than 1\% of the target stars, and are ignored in what follows.
	}
\end{figure}
\begin{figure}[!tb]
	\includegraphics{figures/fig17_radius_on_x.pdf}
	\label{fig:fig17_radius_on_x}
	\caption{Same as Fig.~\protect\ref{fig:fig17_replica}, but as a function of 
	stellar radius. $1/R_\star^2$ selection weight is visible as the envelope 
	of the upper subplot, and the same outliers are present.
	}
\end{figure}


We evaluate \texttt{Merit} for all the star systems in our modified
TRILEGAL catalog, and then choose the highest-\texttt{Merit} 
$2\times10^5$ as target
stars to be observed at 2 minute cadence.  Target stars selected in
this manner are shown in Fig.~\ref{fig:positions_pointings}. 
This statistic is simpler than the procedure outlined in Section 6.7
of~\citetalias{Sullivan_2015} and it produces a nearly identical
population of target stars (shown in Fig~\ref{fig:fig17_replica}).
Our approach for full frame image simulation is different from that
of~\citetalias{Sullivan_2015}, and we describe it below.

We generalize our \texttt{Merit} statistic to Extended Missions as 
follows: over an
entire mission, the total number of observations a star receives is
the sum of its observations in the Primary and Extended Missions:
$N_\text{obs}=N_\text{primary}+N_\text{extended}$.  If
$N_\text{extended}=0$ for a given star, then do not select that star as a
target star in the Extended Mission.  Else, compute its \texttt{Merit}
(Eq.~\ref{eq:merit}) substituting
$N_\text{obs}=N_\text{primary}+N_\text{extended}$.  In this manner
stars that are observed more during the Primary Mission are more
likely to be selected during the Extended Mission.
 

\paragraph{Alternative prioritization approaches:}

It is worth emphasizing that our scheme for selecting target stars for
an Extended Mission does not make use of any information on whether
candidate transit events were detected during the Primary Mission.  If
a star were observed at short cadence for an entire year, and no
candidate events were found, it might be sensible to disregard that
star in the Extended Mission in favor of stars that have never been
observed at short cadence -- particularly those with candidate events
that were detected in the Primary Mission full-frame images.  These
and related concerns are discussed further in the accompanying
document\textsuperscript{\ref{fn:wiki}}.

More abstractly, the procedure of simply applying Eq.~\ref{eq:merit}
attempts to select a stellar sample that will yield the most small
transiting planets around the brightest stars.  An alternative
approach would be to select stars that will give the most 
\textit{relative benefit} in 2 minute postage stamps over 30 
minute cadence observations since all stars will be present in the
30 minute images.  This `relative benefit' could be based on
improvement in transit detectability, or perhaps improved 
capacity to resolve the partial transit phases.

For purposes of transit detection, the difference between 2 and 30
minute cadence matters most when transits have short durations -- in
other words for small stars, and for close-in planets.  Switching to
this alternative approach would consequently bias us even more
strongly toward selecting M dwarfs.  We already select almost every M
dwarf with $I_c < 14$.  The limiting $I_c$ magnitude for detecting
$R_p > 4R_\oplus$ planets with \tess is $\sim\!16$, which is where we
see the dimmest stars in Fig.~\ref{fig:fig17_replica}.

Additionally, the procedure of applying Eq.~\ref{eq:merit} and
assuming that it will maximize the number of small planets that \tess
will detect about bright stars ignores the dependence of
planet occurrence rates on stellar properties, or on the properties of
 planets already known to exist around those stars. If the goal is literally to maximize the number of planet
detections (a goal which we do not advocate) then \tesss target selection might take these dependences into
account. For example the planet occurrence rates measured with {\it Kepler} data could be used
to upweight those types of stars most likely to have planets.
Likewise~\protect\citet{kipping_transit_2016} note that the
probability for a star with short-period transiting planets to have additional transiting
planets is dependent upon the radii and periods of the
short-period planets.  They navigate the optimization problem using artificial neural
networks (ANNs) trained to select for features that improve the
probability of detecting transiting outer companions.  \tess might
benefit from a similar approach in target selection.

\paragraph{Alternative prioritization approaches in Extended Missions:}
Our \texttt{Merit} statistic also neglects the possibility of an Extended
Mission which only observes stars with known planets or planet
candidates (\tesss objects of interest, or those from other transit
and RV surveys) at short cadence.  This approach would free up a
considerable portion of \tesss data mass for full frame images at
\textit{e.g.}, 15 minutes rather than the current nominal 30 minutes.

\paragraph{Approach to full-frame images:}
\label{sec:FFI_simulation}
We want to simulate the full frame image detections in a
computationally tractable manner.  While~\citetalias{Sullivan_2015}
evaluated the phase-folded SNR for every potentially transiting object
about each of the $\sim\!1.6\times10^8$ stars in our synthetic catalog,
we consider only the stars for which \tess could plausibly detect a
sub-Neptune planet during a 3-year mission.  Most stars that
\tess sees are too dim or too large to detect $R_p<4R_\oplus$ planets
-- while we expect many giant planet detections towards the galactic
plane (\citetalias{Sullivan_2015} Fig 19), small-planet detections are
more nearly isotropic, since practically all occur for stars at
$<1\rm\ kpc$.  While thousands of transiting giant planets will be revealed by \tess,
we assume here that the prospects for detecting smaller planets are more likely
to help discriminate between different scenarios for the Extended Mission.

In this vein, we only simulate full frame image detections for the
$3.8\times10^6$ highest \texttt{Merit} stars following the
$2\times10^5$ highest \texttt{Merit} stars that are selected for 2-minute observations.
This number ($3.8\times10^6$) was initially estimated based
on the number of searchable stars about which we expect \tess to be
able to detect sub-Neptune radius
planets~\citep{winn_searchable_2013}. Our model for the detection process for the FFI data is 
identical to that of the PS data, except that the effects of time-smearing on
the apparent transit duration and depth are taken into account (which is
important for durations $\lesssim 1$~hour).

To check that the $4\times10^6$ highest \texttt{Merit} stars includes 
all stars about which \tess might detect sub-Neptune radius planets, 
we repeated this process for the Primary Mission using $5.8\times10^6$,
$9.8\times10^6$ and $19.8\times10^6$ `full frame image' stars, and
confirmed that there was no significant difference in the planet
yields at $R_p\le4R_\oplus$ between any of the cases.
%see ext_sim_notes/160817_on_the_FFI_assumtion. (10 Monte Carlo realizations each)
With an increased number of FFI stars, the number of detected giant planets also increased,
particularly near the galactic
disk.  Meanwhile the number of sub-Neptune planets remained
fixed. This convinced us that $4\times10^6$ is a sufficient
number of target stars to compute detection statistics for sub-Neptune planets.
