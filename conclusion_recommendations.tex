\section{Concluding remarks and recommendations}
\label{sec:conclusions}

This trade-study laid out technical requirements, science goals, opportunities, and risks relevant to observing with \tess after its first two years.
The baseline science requirements for \tesss Primary Mission are defined~\citep{ricker_transiting_2014}, and an Extended Mission offers a chance to re-prioritize.
Is securing \tesss ephemerides more important than probing out to the longest period planets? 
Are the variations in absolute new planet yields, at $\lesssim30\%$ between all the Extended Missions, more valuable than other desires and opportunities?

We leave these questions open for the community to discuss at the \tess Wiki\footnote{\url{https://spacebook.mit.edu/display/TESS/Extended+Missions}. Contact \url{luke@astro.princeton.edu}, copying \url{imeister@mit.edu}, for an account with editing rights.}.


\subsection{Recommendations}
\label{sec:recommendations}
\begin{description}
  
	\item[Analyze target prioritization problem] and decide whether or not to optimize target selection based on results. %for instance \'a la~\protect\citet{kipping_transit_2016}. 
	Planet occurrence rates are functionally dependent on a star's properties -- should this affect which stars we give upgraded cadence?
	Is our proposed \texttt{Merit} statistic (Eq.~\ref{eq:merit}) how we want to prioritize targets?
	
      \item[Optimize cadence:] observing $4\times10^5$ target stars at 
	      4 minute cadence, rather than $2\times10^5$ at 2 minute cadence,
	      may improve prospects for transit detection. 
	      Ideally a metric would be devised that
        quantifies, for each star, how much {\it improvement} would
        result from observing at short-cadence versus long-cadence.
        We could then prioritize the stars for which short-cadence 
        delivers the most significant benefit. 
        For instance if the number of short-cadence stars could be 
        greatly reduced with little effect on the planet detection
        statistics, then this would allow the FFIs to be returned at a
        higher cadence, which would likely be desirable.
	
	\item[Take steps to address the `upgrading cadence' problem:]
          if there is a likely transiting planet in full frame image
          data, upgrading the planet to short cadence in future
          observing sectors improves the probability and fidelity of
          detection.
	
	\item[Guest Investigator Office or \tess Science Office should
          solicit advice] from experts in asteroseismology and
          variable-sky astronomy to understand how Extended Missions
          affect their science cases (\textit{e.g.}, data throughput
          rates are particularly important for time-sensitive
          supernovae observations).  More broadly, solicit community
          feedback during the process of defining the Extended
          mission.  This may entail a call for white papers, comments
          on this report, or direct proposals to the GI office.  As
          exemplified in NASA’s 2016 Astrophysics Senior Review,
          everyone benefits from the discussions generated by such
          community feedback~\citep{donahue_senior_2016}.
	
	\item[Decide (explicitly or implicitly) on weights between our
          proposed metrics.]  Perhaps also brainstorm others --
          Sec.~\ref{sec:comparing_pointing_strategies} gives a
          summarized list.  Are the $\lesssim30\%$ variations in
          absolute new planet yields more valuable than other desires
          or opportunities?
	
	\item[Simulate combining \tess and \ktwo data from
          \rm{\elong\:} \textit{and} \rm{\eshort}.]  This is perhaps
          the most important qualitative difference between observing
          towards and away from the ecliptic.  Would \tess\!+\ktwo
          enable more discoveries out at long periods than
          alternatives?  How many of the new planets that \tess
          detects on the ecliptic will actually be detected by \ktwo?
          Is the value-added of combining datasets a compelling case
          compared with discovery?
	
	\item[Further topics for study:]
	\textit{1.)}
	After how many years would the planet yield start to show steeply diminishing returns?
	Suppose, for example, the Primary Mission were repeated indefinitely.
  How much dimmer would the host stars of new planets be, in each year?
	Would the planets tend to be smaller?
	Would there be fewer of them?
	\textit{2.)}
	Study the limiting case of \hemis\:for 2 years, vs Primary Mission repeat for 2 years, vs \npole$\times$2 for 2 years. The most interesting aspect here is the continuous viewing zones: 2 CVZs at 14-day sampling, vs. 2 CVZs at continuous sampling for one-year each, vs. 1 CVZ for full sampling over two years.\footnote{Our simplified perfect period extraction would miss challenges from the data processing.}
\end{description}
