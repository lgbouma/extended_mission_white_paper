\section{Concluding remarks and recommendations}
\label{sec:conclusions}

Although the science requirements for \tesss Primary Mission have already been
written~\citep{ricker_transiting_2014}, an Extended Mission offers the
entire astronomical community a chance to rethink and reprioritize the use of the spacecraft.
A \tess Extended Mission would have wide-ranging applications, and should be 
planned and proposed with much broader goals than exoplanet science alone. 
The purpose of this trade study was to inform the exoplanet-related portion of 
that discussion. By simulating different scenarios for a third year of 
operations, we can anticipate
the numbers and types of planets that would be discovered. Our study has also highlighted other issues,
such as the 'stale ephemerides' problem, and the opportunity to change the
selection of the PSs and the cadence of the FFIs for diverse reasons.
This study was not designed to provide any definitive answers; but rather,
to provide materials to use in further discussions.  Our study has also
prompted us to make
some recommendations for future work that would be useful in this regard.

\subsection{Recommendations}
\label{sec:recommendations}
\begin{description}
  
	\item[Deeper analysis of the target prioritization scheme.] How should the
	results of the Primary Mission, or other sources of new data, be used to
	choose target stars for finer time sampling during an Extended Mission?
	
    \item[Optimize cadence.] What is the optimal time sampling for transit 
    detection? For example, observing $2\times10^5$ target stars at 
    4-minute cadence, rather than at 2-minute cadence,
    would allow the FFIs to be returned more frequently.
    This could in turn improve prospects for transit detection.  
    (Preliminary numerical experiments indicate that this is indeed the 
    case.)
	      
	      In addition, a metric should be devised that
        quantifies, for each star, how much {\it improvement} would
        result from observing at a shorter cadence.
        Stars could then be prioritized according to this improvement
	statistic, rather than an overall planet-detectability statistic.
        For instance if the number of short-cadence stars could be 
        greatly reduced with little effect on the planet detection
        statistics, then this might allow the FFIs to be returned at a
        higher cadence.
	
	\item[Take steps to address the `upgrading cadence' problem:]
          if there is a likely transiting planet in full frame image
          data, upgrading the planet to short cadence in future
          observing sectors improves the probability and fidelity of
          detection. How is this being addressed for the Primary Mission?
          Could this be an argument to observe the southern
          sky in year 3, in order to have extra time to prepare?
	
	\item[Solicit expert advice] from experts in asteroseismology, transient detection,
	and other relevant areas to understand how the parameters of an Extended Mission
          would affect their scientific prospects. Comments on this report sent to
	  the authors will be gratefully received. A more formal and comprehensive process would
          be a call for White Papers organized
	  by the Guest Investigator program or the \tess Science Office. As
          exemplified in NASA’s 2016 Astrophysics Senior Review,
          everyone benefits from the discussions generated by such
          community feedback~\citep{donahue_senior_2016}.
	
	\item[Consider the relative importance of our proposed exoplanet-detection 
	metrics.]
	Sec.~\ref{sec:comparing_pointing_strategies} summarizes these; the reader 
	may have others to suggest. The crude metric of "total number of new planet 
	detections" is not
	likely to be the most important, and we have found that the scenarios we considered
	differ by $\lesssim30\%$ in
          this regard. 
	
	\item[Simulate combining \tess and \ktwo data from
          \rm{\elong\:} \textit{and} \rm{\eshort}.]  This is perhaps
          the most important qualitative difference between observing
          towards and away from the ecliptic.  Would \tess\!+\ktwo
          enable more discoveries out at long periods than
          alternatives?  How many of the new planets that \tess
          detects on the ecliptic will actually be detected by \ktwo?
          Is the value-added of combining datasets a compelling case
          compared with discovery?
          
    \item[Ensure {\rm \npole} adequately mitigates scattered sunlight.]
    See description in Sec.~\ref{sec:proposed_pointings}.
	
    \item[Point of diminishing returns.] Suppose, for example, the Primary Mission were repeated indefinitely.
    At what point would the planet discovery rate start falling significantly below that of Years 1-3?
    How much fainter would the host stars of new planets be, in each year?
	
\end{description}
