\documentclass{article}
\usepackage{amssymb}
\begin{document}

Tagline: The views and opinions expressed in this white paper are those of the authors and do not necessarily reflect the views of NASA or the \tess Science Team.

The Transiting Exoplanet Survey Satellite (\textit{TESS}) will perform
a two-year survey of nearly the entire sky, with the main goal of detecting
exoplanets smaller than Neptune around bright and nearby stars. There do
not appear to be any fundamental obstacles to continuing science
operations for at least several years after the two-year Primary Mission.
To provide a head start to those who will plan and propose for such a 
mission, we present simulations of exoplanet detections in a third year of \textit{TESS}
operations. Our goal to provide a helpful reference for the exoplanet-related
aspects of any Extended Mission, while recognizing this will be only one part of
a larger community-based discussion of the scientific goals of such a mission.
We use Monte Carlo simulations to try and anticipate the quantities and types
of planets that would be detected in each of 6 plausible scenarios for a
one-year Extended Mission following the two-year Primary Mission.
We find that: (1) the quantity of newly detected sub-Neptune planets
planets does not depend strongly on the schedule of pointings; (2) There is no
sharp fall-off in the planet discovery rate in the third year; (3) An important
function of an Extended Mission would be improving our ability to predict the times of future
transits and occultations of {\it TESS}-detected planets.
\end{document}
