\documentclass{article}
\usepackage{amssymb}
\begin{document}
The Transiting Exoplanet Survey Satellite (\textit{TESS}) will perform a
two-year, nearly all-sky survey for transiting exoplanets.  Any Extended
Mission will likely need to be organized while the Primary Mission is occupying
most of the \textit{TESS} team's attention.  To provide a head start to those
who are planning and proposing for an Extended Mission, this white paper
presents some freelance considerations on planet detection.

Using Monte Carlo simulations, we numerically compute the quantities and types
of planets that would be detected during several plausible scenarios for a
one-year Extended Mission following the two-year Primary Mission. Our main
focus is on strategies for scanning the sky, for which we consider six
distinct scenarios (see executive summary). 

We find that: \textit{1)} The overall quantity of detected $R_p<4R_\oplus$
planets does not depend strongly on the sky-scanning schedule.  Among the
scenarios we consider, the number of newly-detected planets with radii less
than $4R_\oplus$ is the same to within about 30\%.  \textit{2)} There is no
sharp fall-off in the planet discovery rate in Year 3; the number of
newly-detected $R_p \lesssim 4R_\oplus$ planets in Year 3
is approximately the same as the number detected in either of the first two
years.
\textit{3)} Apart from detecting new planets, a potentially important function
of an Extended Mission is to improve our ability to predict the times of future
transits and occultations of {\it TESS}-detected planets.  With data from the
Primary Mission alone, the uncertainty in planetary orbital periods will
inhibit follow-up observations after only a few years, as the transit
ephemerides become stale. By re-observing the same sky that was observed in the
Primary Mission, certain Extended Mission sky-scanning strategies can address
this issue.
\end{document}