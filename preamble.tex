\documentclass{tufte-handout}
%\geometry{showframe}% for debugging purposes -- displays the margins
\usepackage[T1]{fontenc}
\usepackage[utf8]{inputenc}
\usepackage{float} % has option [H] for force placement
\usepackage{graphicx} % useful for eg. pdf or eps or svg graphics
\usepackage[space]{grffile}
\usepackage{latexsym}
\usepackage{textcomp}
\usepackage{cleveref} % \cref{} references multiple sections separated by commas
\usepackage{longtable}
\usepackage{multirow,booktabs}
\usepackage{amsfonts,amsmath,amssymb}
\usepackage{natbib}
\usepackage{url}
\usepackage{hyperref}
\usepackage{wasysym}
\usepackage{lscape} % for landscape
\usepackage{pdfpages}
\usepackage{verbatim} % for \begin{comment}
\usepackage{todonotes}
\usepackage[english]{babel}
\usepackage{longtable} % For caption on head of table, long-table example.
\usepackage{afterpage}  % to put at beginning of longtables to float them
\usepackage[toc,page,title,titletoc]{appendix} % For appendices
%\usepackage[title,titletoc]{appendix}
\usepackage{tocloft}
\hypersetup{colorlinks=false,pdfborder={0 0 0}}
% You can conditionalize code for latexml or normal latex using this.
\newif\iflatexml\latexmlfalse
\providecommand{\tightlist}{\setlength{\itemsep}{0pt}\setlength{\parskip}{0pt}}%

\setkeys{Gin}{width=\linewidth,totalheight=\textheight,keepaspectratio}
\graphicspath{{graphics/}}

% The units package provides nice, non-stacked fractions and better spacing
% for units.
\usepackage{units}
% The fancyvrb package lets us customize the formatting of verbatim
% environments.  We use a slightly smaller font.
\usepackage{fancyvrb}
\fvset{fontsize=\normalsize}
% Small sections of multiple columns
\usepackage{multicol}
% These commands are used to pretty-print LaTeX commands
\newcommand{\doccmd}[1]{\texttt{\textbackslash#1}}% command name -- adds backslash automatically
\newcommand{\docopt}[1]{\ensuremath{\langle}\textrm{\textit{#1}}\ensuremath{\rangle}}% optional command argument
\newcommand{\docarg}[1]{\textrm{\textit{#1}}}% (required) command argument
\newenvironment{docspec}{\begin{quote}\noindent}{\end{quote}}% command specification environment
\newcommand{\docenv}[1]{\textsf{#1}}% environment name
\newcommand{\docpkg}[1]{\texttt{#1}}% package name
\newcommand{\doccls}[1]{\texttt{#1}}% document class name
\newcommand{\docclsopt}[1]{\texttt{#1}}% document class option name

\setcounter{secnumdepth}{3} % show section and subsection numbers
%\bibliographystyle{plainnat} % plain numbered bibliography style
\bibliographystyle{abbrvnat} % abbreviated style
% In align* environment, use \numberthis to only number last eqn
\newcommand\numberthis{\addtocounter{equation}{1}\tag{\theequation}}
% For footnote memory


% % TITLE, AUTHORS

\title{Planet Detection Simulations for Several Possible TESS Extended Missions}
%N.b. \thanks{} in author gives a footnote
%\renewcommand\Authfont{\itshaped\large}
%\renewcommand\Affilfont{\itshape\tiny}

\author{\normalsize{Luke Bouma$^{1,2,3}$, 
	Josh Winn$^{1,2,3}$, 
	Jacobi Kosiarek$^{2}$ 
	and Peter McCullough$^{4,5}$}}

\date{}

%\date{24 January 2009}  % if left out, the current date will be used
