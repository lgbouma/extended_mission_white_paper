\section*{Acknowledgements}
We thank Peter Sullivan, for his support in the early phases of this 
work and for helpful comments on our results;
Roland Vanderspek, for fielding numerous questions concerning \tess hardware; 
Jack Lissauer, for input regarding the value of observing \keplers field with 
\textit{TESS};
and Tim Morton, for emphasizing the importance of the false positive problem.
We thank the many other participants in informal discussions on this subject over the last few years.
More broadly we are grateful to the hardworking teams 
at MIT, NASA Goddard, NASA Ames, CfA, STScI, Orbital Sciences, and other \textit{TESS} partner institutions.
We thank the \tess project for providing the computing resources used in this work, and Ed 
Morgan, Isaac Meister, and Kenton Philips for keeping those machines running.

\vspace{0.5cm}
\textit{Facilities}: \tess\!, \kepler

\textit{Software}: matplotlib~\citep{hunter_matplotlib_2007}, NumPy~\citep{walt_numpy_2011}, SciPy~\citep{jones_scipy_2001}, pandas~\citep{mckinneypandas}, JPL NAIF's SPICE library~\citep{acton_SPICE_1996}, and the IDL Astronomy User's Library~\citep{landsman_idl_1995}.

\textit{Resources}: This research has made use of the NASA Astrophysics Data System and the NASA Exoplanet Archive. The NASA Exoplanet Archive is operated by the California Institute of Technology, under contract with the National Aeronautics and Space Administration under the Exoplanet Exploration Program.
This paper also makes use of data collected by the Kepler mission. Funding for the Kepler mission is provided by the NASA Science Mission directorate.
